% !TeX encoding = UTF-8
% !TeX spellcheck = it_IT
% !TeX root = funzioni.tex
% 7/11/2017 :: 10:11:07 :: 
\chapter{Limiti}
\section{Limite infinito per x che tende a valore finito}
\begin{esempiot}{Razionale fratta *}{}
	Consideriamo la funzione\[f(x)=\dfrac{1}{x}\]
\end{esempiot}
\begin{enumerate}[noitemsep]
	\litem{Classificazione}la funzione è una funzione razionale fratta\index{Funzione!razionale!fratta}.
	\litem{Dominio}il dominio della funzione è $\R-\lbrace 0\rbrace$.
	\litem{Positività}risolviamo la disequazione
	\begin{align*}
	&\dfrac{1}{x}>0 
	\end{align*} La funzione è positiva per 
	Ottengo il~\cref{fig:limite1}. La funzione è positiva per $x>0$
	\item Asintoti La funzione non è definita per $x=0$ e studiando la positività abbiamo \begin{align*}
	\lim_{x\to x_0^+}\dfrac{1}{x}=&+\infty\\
	\lim_{x\to x_0^-}\dfrac{1}{x}=&-\infty
	\end{align*}\index{Asintoto!verticale}
\end{enumerate}
La~\cref{exa:ratio2} riassume quanto detto.
\begin{figure}
	\captionsetup{name=Grafico}
	\centering
	\includestandalone[width=8.5cm]{dominio/disequazione3}
	\caption{Segno funzione}
	\label[graf]{fig:limite1}
\end{figure}
\begin{funzionet}{Razionale fratta *}{limite1}
	\includestandalone[width=\textwidth]{dominio/razio2}
	\tcblower
	\begin{itemize}
		\item $y=\dfrac{1}{x}$
		\item Dominio $\R-\lbrace0\rbrace$
		\item Codominio $\R$
		\item Positività $x\geq0$
	\end{itemize}
\end{funzionet}